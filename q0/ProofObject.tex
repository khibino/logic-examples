\documentclass[cjk,dvipdfm,14pt]{beamer}
\usetheme{Hibi}
\usefonttheme{professionalfonts}
\usepackage{packages}

\lstset{language=Haskell,basicstyle=\ttfamily}

\title{構成的プログラムと型}
%% \subtitle{--  --}
\author{日比野 啓}
\date{2013-05-29}

\begin{document}

\begin{frame}
\maketitle
\end{frame}

\begin{frame}[fragile]
\frametitle{本日の概要}

\begin{itemize}
\item 型とプログラム
\item 関数と構成的プログラム
\item 値と型付け
\item 論理パズルの例
\end{itemize}

\end{frame}

\begin{frame}[fragile]
質問はいつでもどうぞ
\end{frame}

\begin{frame}[fragile]
\frametitle{本日の概要}

\begin{itemize}
\item { \color{red} 型とプログラム }
\item 関数と構成的プログラム
\item 値と型付け
\item 論理パズルの例
\end{itemize}

\end{frame}

\begin{frame}[fragile]
\frametitle{型とプログラム}
型とは\verb|?|
\begin{itemize}
\item { \color{red} 型とはその型を持つ値の性質を示すもの }
\end{itemize}
\end{frame}

\begin{frame}[fragile]
\frametitle{型とプログラム}

型とは\verb|?|
\begin{itemize}
\item 型とはその型を持つ値の性質を示すもの
\end{itemize}

プログラムとは\verb|?|
\begin{itemize}
\item { \color{red} 目的の値を作り出すもの }
\end{itemize}

\end{frame}

\begin{frame}[fragile]
\frametitle{型とプログラム}

型とは\verb|?|
\begin{itemize}
\item 型とはその型を持つ値の性質を示すもの
\end{itemize}

プログラムとは\verb|?|
\begin{itemize}
\item 目的の値を作り出すもの
\end{itemize}

正しいプログラムは\\正しく型が付いた値を作り出す
\begin{itemize}
\item { \color{red} 正しいプログラムは作り出した値の性質を証明する }
\end{itemize}

\end{frame}

\begin{frame}[fragile]
\frametitle{型とプログラム - 例 - Even0.hs}

値を検査し、\\
それが正しい場合においてのみ Even を作る
\begin{lstlisting}
-- data Maybe a = Just a
--              | Nothing

data Even = Even' Int

mayEven :: Int -> Maybe Even
mayEven i
  | i `rem` 2 == 0 = Just (Even' i)
  | otherwise      = Nothing
-- 例えば Even' は他からは隠蔽する
\end{lstlisting}

\end{frame}

\begin{frame}[fragile]
※ 画面切り替え練習
\end{frame}

\begin{frame}[fragile]
\frametitle{実習}

\begin{lstlisting}
 % ghci Even0.hs
 ...
 > mayEven 10
 ...
 > mayEven 5
\end{lstlisting}
※ Tab で補完できる

\end{frame}

\begin{frame}[fragile]
\frametitle{型とプログラム - 例 - Even1.hs}

検査しなくてもEvenを作るくことができる公理系の例
\begin{lstlisting}
data Even = Double Int

double :: Int -> Even
double i = Double i

mayEven :: Int -> Maybe Even
mayEven i
  | rm == 0   = Just (double qt)
  | otherwise = Nothing
  where (qt, rm) = i `quotRem` 2
\end{lstlisting}

\end{frame}

\begin{frame}[fragile]
\frametitle{実習}

\begin{lstlisting}
 % ghci Even1.hs
 ...
 > double 5
 ...
 > mayEven 10
 ...
 > mayEven 5
\end{lstlisting}

\end{frame}

\begin{frame}[fragile]
\frametitle{型とプログラム - まとめ}

\begin{itemize}
\item 型とはその型を持つ値の性質を示すもの
\item プログラムとは目的の値を作り出すもの
\item 正しいプログラムは正しく型が付いた値を作り出し、その性質を証明する
\end{itemize}

\end{frame}

\begin{frame}[fragile]
\frametitle{本日の概要}

\begin{itemize}
\item 型とプログラム
\item { \color{red} 関数と構成的プログラム }
\item 値と型付け
\item 論理パズルの例
\end{itemize}

\end{frame}

\begin{frame}[fragile]
\frametitle{関数と構成的プログラム}

関数とは\verb|?|
\begin{itemize}
\item { \color{red} 値の持つ性質から値の持つ性質を導き出す}
\end{itemize}

\end{frame}

\begin{frame}[fragile]
\frametitle{関数と構成的プログラム}

関数とは\verb|?|
\begin{itemize}
\item 値の持つ性質から値の持つ性質を導き出す
\item { \color{red} 関数は証明における推論規則 }
\end{itemize}

\end{frame}

\begin{frame}[fragile]
\frametitle{関数と構成的プログラム}

関数とは\verb|?|
\begin{itemize}
\item 値の持つ性質から値の持つ性質を導き出す
\item 関数は証明における推論規則
\end{itemize}

{ \color{red} プログラムは推論規則の組み合わせで証明を構成する }

\end{frame}

\begin{frame}[fragile]
\frametitle{関数 - 例 - Even2.hs}

さらに別の公理系を考える
\begin{lstlisting}
data EvenPrime = Even' Int

zero' :: EvenPrime
zero' =  Even' 0

plus2' :: EvenPrime -> EvenPrime
plus2' (Even' i) = Even' (i + 2)
\end{lstlisting}

\end{frame}

\begin{frame}[fragile]
\frametitle{関数 - 例 - Even3.hs}

拡張すると、Evenを作り出す加算ができるように
\begin{lstlisting}
data Even = Even
            (EvenPrime -> EvenPrime)

zero :: Even
zero =  Even id

two :: Even
two =  Even plus2'

(<+>) :: Even -> Even -> Even
(Even f) <+> (Even g) = Even (f . g)
\end{lstlisting}

\end{frame}

\begin{frame}[fragile]
\frametitle{実習}

\begin{lstlisting}
 % ghci useEven3.hs
 > zero'
 > plus2' (plus2' (plus2' zero'))
 > zero
 > two
 > four
 > six
 > four <+> six
\end{lstlisting}

\end{frame}

\begin{frame}[fragile]
\frametitle{関数と構成的プログラム \\- コード例および実習まとめ}

\begin{enumerate}
\item 0 と 2 の繰り返し加算による公理系 (EvenPrime) を定義した
\item EvenPrime を拡張した Even の定義から、Even と Even の加算が Even になることを導出できた
\end{enumerate}

\end{frame}

\begin{frame}[fragile]
\frametitle{関数と構成的プログラム - まとめ}

\begin{itemize}
\item 関数とは関数は証明における推論規則
\item プログラムは推論規則(関数)の組み合わせで証明を構成する
\end{itemize}

\end{frame}

\begin{frame}[fragile]
質問
\end{frame}

\begin{frame}[fragile]
\frametitle{本日の概要}

\begin{itemize}
\item 型とプログラム
\item 関数と構成的プログラム
\item { \color{red} 値と型付け }
\item 論理パズルの例
\end{itemize}

\end{frame}

\begin{frame}[fragile]
\frametitle{値と型付け}

型とはその型を持つ値の性質を示すもの
\begin{itemize}
\item 値の性質 $\fallingdotseq $ 型の意味\\
 → { \color{red} 値の性質 $\leqq $ 型の意味 }
\end{itemize}

\end{frame}

\begin{frame}[fragile]
\frametitle{値と型付け}

型とはその型を持つ値の性質を示すもの
\begin{itemize}
\item 値の性質 $\fallingdotseq $ 型の意味\\
 → 値の性質 $\leqq $ 型の意味
\end{itemize}
{ \color{red} 幽霊型 - Phantom Type }

\end{frame}

\begin{frame}[fragile]
\frametitle{値と型付け - 幽霊型 - 例 - Maybe.hs}

\begin{lstlisting}
noInt :: Maybe Int
noInt =  Nothing

noString :: Maybe String
noString =  Nothing

-- noInt == noString
\end{lstlisting}

\end{frame}

\begin{frame}[fragile]
\frametitle{実習}

\begin{lstlisting}
 % ghci Maybe.hs
 > justOne
 > justHello
 > noInt
 > noString
 > justOne == justOne
 > noInt == noString
\end{lstlisting}

\end{frame}

\begin{frame}[fragile]
\frametitle{値と型付け - 幽霊型 - 例 - IntUnit.hs}

\begin{lstlisting}
data Int' u  = Int' Int

(<+>) :: Int' u -> Int' u -> Int' u
Int' a <+> Int' b = Int' (a + b)

data YenUnit  = Yen
type Yen  = Int' YenUnit

data GramUnit = Gram
type Gram = Int' GramUnit
\end{lstlisting}

\end{frame}

\begin{frame}[fragile]
\frametitle{実習}

\begin{lstlisting}
 % ghci IntUnit.hs
 > twentyYen
 > :t twentyYen
 > thirtyGram
 > :t thirtyGram
 > twentyYen <+> twentyYen
 > twentyYen <+> thirtyGram
\end{lstlisting}

\end{frame}

\begin{frame}[fragile]
\frametitle{実習}

\begin{lstlisting}
 > :i Yen
 > :t impose
 > :i Imposed
 > :t impose twentyYen
 > :t twentyImposed
 > twentyImposed <+> twentyImposed
 > twentyYen <+> twentyImposed
\end{lstlisting}

\end{frame}

\begin{frame}[fragile]
\frametitle{値と型付け - まとめ}

値の性質よりも\\
意味の強い型を付けることができる
\begin{itemize}
\item そのような例としては幽霊型がある
\end{itemize}

\end{frame}

\begin{frame}[fragile]
質問
\end{frame}

\begin{frame}[fragile]
\frametitle{本日の概要}

\begin{itemize}
\item 型とプログラム
\item 関数と構成的プログラム
\item 値と型付け
\item { \color{red} 論理パズルの例 }
\end{itemize}

\end{frame}

\begin{frame}[fragile]
\frametitle{論理パズル}

http://ja.wikipedia.org/wiki/ロジックパズル
から持ってきた問題
\\ \hrulefill \\
{ \small
3人のそれぞれの発言から、それぞれの今日の昼食を当ててください。ただし、カレーライス、ラーメン、そばのうちから3人とも別々のものを食べました。
\begin{description}
\item[トンキチ] …。
\item[チンペイ] あいつみたいにそばだったら僕は足りないな。
\item[カンタ] 僕はカレーライスもそばも嫌いなんだ。
\end{description}
}

\end{frame}

\begin{frame}[fragile]
\frametitle{論理パズル - 力技の解法 - L0.hs}

\begin{lstlisting}
 % ghci Even4.hs
 ...
 > good
\end{lstlisting}

\end{frame}

\begin{frame}[fragile]
プログラムの解説 - L0.hs
\end{frame}

\begin{frame}[fragile]
\frametitle{論理パズル - 推論規則を利用した解法 - solve.hs}

\begin{lstlisting}
 % ghci solve.hs
 ...
 > solve
 > :t solve
\end{lstlisting}

\end{frame}

\begin{frame}[fragile]
プログラムの解説 - L1.hs, solve.hs
\end{frame}

\begin{frame}[fragile]
\frametitle{論理パズルの例 - まとめ}

\begin{itemize}
\item 幽霊型をつかい公理と推論規則を定義した
\item 公理と推論規則から欲しい解を得るプログラムを記述した
\end{itemize}

\end{frame}

\begin{frame}[fragile]
質問
\end{frame}

\begin{frame}[fragile]
\frametitle{参考文献}

\begin{itemize}
\item Software Foundations http://www.cis.upenn.edu/~bcpierce/sf/
\item Software Foundations (日本語訳) http://github.com/sfja/sfja
\end{itemize}

\end{frame}

%% \begin{frame}[fragile]
%% \frametitle{}
%% \end{frame}

%% \begin{frame}[fragile]
%% \frametitle{}
%% \end{frame}

\end{document}
