\documentclass[cjk,dvipdfm,14pt]{beamer}
\usetheme{Hibi}
\usefonttheme{professionalfonts}
\usepackage{packages}

\lstset{language=Haskell,basicstyle=\ttfamily}

\title{構成的プログラムと型}
%% \subtitle{--  --}
\author{日比野 啓}
\date{2013-05-29}

\begin{document}

\begin{frame}
\maketitle
\end{frame}

\begin{frame}[fragile]
\frametitle{アジェンダ}

\begin{itemize}
\item プログラムと型
\item ?
\end{itemize}

\end{frame}

\begin{frame}[fragile]
\frametitle{アジェンダ}

\begin{itemize}
\item { \color{red} プログラムと型 }
\item ?
\end{itemize}

\end{frame}

\begin{frame}[fragile]
\frametitle{型とプログラム}
あらためて型とは\verb|?|
\end{frame}

\begin{frame}[fragile]
\frametitle{型とプログラム}
あらためて型とは\verb|?|
\begin{itemize}
\item { \color{red} 型とはその型を持つ値の性質を示すもの }
\end{itemize}
\end{frame}

\begin{frame}[fragile]
\frametitle{型とプログラム}

あらためて型とは\verb|?|
\begin{itemize}
\item 型とはその型を持つ値の性質を示すもの
\end{itemize}

プログラムとは\verb|?|

\end{frame}

\begin{frame}[fragile]
\frametitle{型とプログラム}

あらためて型とは\verb|?|
\begin{itemize}
\item 型とはその型を持つ値の性質を示すもの
\end{itemize}

プログラムとは\verb|?|
\begin{itemize}
\item { \color{red} 目的の値を作り出すもの }
\end{itemize}

\end{frame}


\begin{frame}[fragile]
\frametitle{型とプログラム}

あらためて型とは\verb|?|
\begin{itemize}
\item 型とはその型を持つ値の性質を示すもの
\end{itemize}

プログラムとは\verb|?|
\begin{itemize}
\item 目的の値を作り出すもの
\end{itemize}

{ \color{red} 正しいプログラムは正しく型が付いた値を作り出す }

\end{frame}

\begin{frame}[fragile]
\frametitle{型とプログラム}

あらためて型とは\verb|?|
\begin{itemize}
\item 型とはその型を持つ値の性質を示すもの
\end{itemize}

プログラムとは\verb|?|
\begin{itemize}
\item 目的の値を作り出すもの
\end{itemize}

正しいプログラムは正しく型が付いた値を作り出す

{ \color{red} 正しいプログラムは作り出した値の性質を証明する }

\end{frame}

\begin{frame}[fragile]
\frametitle{型とプログラム}

関数とは\verb|?|

\end{frame}

\begin{frame}[fragile]
\frametitle{型とプログラム}

関数とは\verb|?|
\begin{itemize}
\item { \color{red} 値の持つ性質から値の持つ性質を導き出す}
\end{itemize}

\end{frame}

\begin{frame}[fragile]
\frametitle{型とプログラム}

関数とは\verb|?|
\begin{itemize}
\item 値の持つ性質から値の持つ性質を導き出す
\end{itemize}

{ \color{red} 関数は証明における推論規則 }

\end{frame}


\begin{frame}[fragile]
\frametitle{型とプログラム}

関数とは\verb|?|
\begin{itemize}
\item 値の持つ性質から値の持つ性質を導き出す
\end{itemize}

関数は証明における推論規則

{ \color{red} プログラムは推論規則の組み合わせで証明を構成する }

\end{frame}

\begin{frame}[fragile]
\frametitle{具体例}


\end{frame}

%% \begin{frame}[fragile]
%% \frametitle{}
%% \end{frame}

%% \begin{frame}[fragile]
%% \frametitle{}
%% \end{frame}

\end{document}
