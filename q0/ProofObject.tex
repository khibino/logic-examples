\documentclass[cjk,dvipdfm,14pt]{beamer}
\usetheme{Hibi}
\usefonttheme{professionalfonts}
\usepackage{packages}

\lstset{language=Haskell,basicstyle=\ttfamily}

\title{構成的プログラムと型}
%% \subtitle{--  --}
\author{日比野 啓}
\date{2013-05-29}

\begin{document}

\begin{frame}
\maketitle
\end{frame}

\begin{frame}[fragile]
\frametitle{本日の概要}

\begin{itemize}
\item 型とプログラム
\item 関数と構成的プログラム
\item 論理パズルの例
\end{itemize}

\end{frame}

\begin{frame}[fragile]
質問は随時受けつけます
\end{frame}

\begin{frame}[fragile]
\frametitle{本日の概要}

\begin{itemize}
\item { \color{red} 型とプログラム }
\item 関数と構成的プログラム
\item 論理パズルの例
\end{itemize}

\end{frame}

\begin{frame}[fragile]
\frametitle{型とプログラム}
あらためて型とは\verb|?|
\begin{itemize}
\item { \color{red} 型とはその型を持つ値の性質を示すもの }
\end{itemize}
\end{frame}

\begin{frame}[fragile]
\frametitle{型とプログラム}

あらためて型とは\verb|?|
\begin{itemize}
\item 型とはその型を持つ値の性質を示すもの
\end{itemize}

プログラムとは\verb|?|
\begin{itemize}
\item { \color{red} 目的の値を作り出すもの }
\end{itemize}

\end{frame}

\begin{frame}[fragile]
\frametitle{型とプログラム}

あらためて型とは\verb|?|
\begin{itemize}
\item 型とはその型を持つ値の性質を示すもの
\end{itemize}

プログラムとは\verb|?|
\begin{itemize}
\item 目的の値を作り出すもの
\end{itemize}

正しいプログラムは\\正しく型が付いた値を作り出す
\begin{itemize}
\item { \color{red} 正しいプログラムは作り出した値の性質を証明する }
\end{itemize}

\end{frame}

\begin{frame}[fragile]
\frametitle{型とプログラム - 例 - Even0.hs}

値を検査し、それが正しい場合においてのみ Even を作る
\begin{lstlisting}
-- data Maybe a = Just a
--              | Nothing

data Even = Even' Int

mayEven :: Int -> Maybe Even
mayEven i
  | i `rem` 2 == 0 = Just (Even' i)
  | otherwise      = Nothing
-- 例えば Even' は他からは隠蔽する
\end{lstlisting}

\end{frame}

\begin{frame}[fragile]
\frametitle{型とプログラム - 例 - Even1.hs}

検査しなくてもEvenを作るくことができる公理系の例
\begin{lstlisting}
data Even = Double Int

double :: Int -> Even
double i = Double i

mayEven :: Int -> Maybe Even
mayEven i
  | rm == 0   = Just (double qt)
  | otherwise = Nothing
  where (qt, rm) = i `quotRem` 2
\end{lstlisting}

\end{frame}

\begin{frame}[fragile]
\frametitle{型とプログラム - まとめ}

\begin{itemize}
\item 型とはその型を持つ値の性質を示すもの
\item プログラムとは目的の値を作り出すもの
\item 正しいプログラムは正しく型が付いた値を作り出し、その性質を証明する
\end{itemize}

\end{frame}

\begin{frame}[fragile]
ここまでで質問とか
\end{frame}

\begin{frame}[fragile]
\frametitle{本日の概要}

\begin{itemize}
\item 型とプログラム
\item { \color{red} 関数と構成的プログラム }
\item 論理パズルの例
\end{itemize}

\end{frame}

\begin{frame}[fragile]
\frametitle{関数}

関数とは\verb|?|
\begin{itemize}
\item { \color{red} 値の持つ性質から値の持つ性質を導き出す}
\end{itemize}

\end{frame}

\begin{frame}[fragile]
\frametitle{関数}

関数とは\verb|?|
\begin{itemize}
\item 値の持つ性質から値の持つ性質を導き出す
\item { \color{red} 関数は証明における推論規則 }
\end{itemize}


\end{frame}

\begin{frame}[fragile]
\frametitle{関数}

関数とは\verb|?|
\begin{itemize}
\item 値の持つ性質から値の持つ性質を導き出す
\item 関数は証明における推論規則
\end{itemize}

{ \color{red} プログラムは推論規則の組み合わせで証明を構成する }

\end{frame}

\begin{frame}[fragile]
\frametitle{関数 - 例 - Even2.hs}

さらに別の公理系を考える
\begin{lstlisting}
data EvenPrime = Even' Int

zero' :: EvenPrime
zero' =  Even' 0

plus2' :: EvenPrime -> EvenPrime
plus2' (Even' i) = Even' (i + 2)
\end{lstlisting}

\end{frame}

\begin{frame}[fragile]
\frametitle{関数 - 例 - Even3.hs}

拡張すると、Evenを作り出す加算ができるように
\begin{lstlisting}
data Even = Even
            (EvenPrime -> EvenPrime)

zero :: Even
zero =  Even id

two :: Even
two =  Even plus2'

(<+>) :: Even -> Even -> Even
(Even f) <+> (Even g) = Even (f . g)
\end{lstlisting}

\end{frame}

\begin{frame}[fragile]
\frametitle{関数 - 実行してみる}

\begin{lstlisting}
 % ghci Even4.hs
\end{lstlisting}

\end{frame}


\begin{frame}[fragile]
\frametitle{関数}

\begin{enumerate}
\item 0 と 2 の繰り返し加算による公理系 (EvenPrime) を定義した
\item EvenPrime を拡張した Even の定義から、Even と Even の加算が Even になることを導出できた
\end{enumerate}

\end{frame}

\begin{frame}[fragile]
質問
\end{frame}

\begin{frame}[fragile]
\frametitle{本日の概要}

\begin{itemize}
\item 型とプログラム
\item 関数と構成的プログラム
\item { \color{red} 論理パズルの例 }
\end{itemize}

\end{frame}

\begin{frame}[fragile]
\frametitle{論理パズル}

http://ja.wikipedia.org/wiki/ロジックパズル

から持ってきた問題

\end{frame}

\begin{frame}[fragile]
\frametitle{論理パズル - 力技の解法 - L0.hs}

\begin{lstlisting}
 % ghci Even4.hs
 ...
 > good
\end{lstlisting}

\end{frame}

\begin{frame}[fragile]
\frametitle{論理パズル - 推論規則を利用した解法 - solve.hs}

\begin{lstlisting}
 % ghci solve.hs
 ...
 > solve
 > :t solve
\end{lstlisting}

\end{frame}

%% \begin{frame}[fragile]
%% \frametitle{}
%% \end{frame}

%% \begin{frame}[fragile]
%% \frametitle{}
%% \end{frame}

\end{document}
